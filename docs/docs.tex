\documentclass[a4paper]{amsart}

%\geometry{showframe}% for debugging purposes -- displays the margins

\usepackage{amsmath}
\usepackage{amssymb}
\usepackage{color}
\usepackage{hyperref}

% Set up the images/graphics package
\usepackage{graphicx}
\setkeys{Gin}{width=\linewidth,totalheight=\textheight,keepaspectratio}

% The fancyvrb package lets us customize the formatting of verbatim
% environments.  We use a slightly smaller font.
\usepackage{fancyvrb}
\fvset{fontsize=\small}

% Syntax highlighting
<< pygments['pastie.tex'] >>

\setlength{\parindent}{0pt}
\setlength{\parskip}{1ex plus 0.5ex minus 0.2ex}

\begin{document}

The OpenGamma R package allows you to connect with a running OpenGamma server. The examples in this document were generated using the OG-Example server and the dummy data it contains.

\section{Running the OG-Example Server}

The OG-Example server must be running first.

TODO - how to install and start.

When the server is running, you should be able to point your browser to \verb|localhost:8080| and you should see something like this:

<< d['take-screenshot.py|idio|l']['load-portfolio-page'] >>

\includegraphics[width=5in]{dexy--screenshot-portfolio-page.pdf}

\section{Securities}

Let's use the web interface to explore the securities that are available:

<< d['take-screenshot.py|idio|l']['load-securities-page'] >>

\includegraphics[width=5in]{dexy--screenshot-securities-page.pdf}

<< d['take-screenshot.py|idio|l']['show-security'] >>

\includegraphics[width=5in]{dexy--screenshot-view-security.pdf}

We can use any of the listed identifiers (along with their scheme) to retrieve a reference to this security, as below:

<< d['time_series.R|fn|idio|rint|pyg|l']['security-amazon-tickers'] >>

Alternatively, we can pass OpenGamma's internal unique identifier like this:

<< d['time_series.R|fn|idio|rint|pyg|l']['security-amazon-uniqueid'] >>

Attributes can be accessed by calling ExpandSecurity or, for more detail, a more specific method such as ExpandEquitySecurity (for an equity). There are also some GetSecurityX methods which return attributes:

<< d['time_series.R|fn|idio|rint|pyg|l']['security-properties'] >>

\section{Time Series}

Let's use the web interface to explore the time series that are available:

<< d['take-screenshot.py|idio|l']['load-timeseries-page'] >>

\includegraphics[width=5in]{dexy--screenshot-timeseries-page.pdf}

We can click on any of the time series listed on the left and explore the data:

<< d['take-screenshot.py|idio|l']['show-timeseries'] >>

\includegraphics[width=5in]{dexy--screenshot-view-timeseries.pdf}

We see that the ticker is \verb|<< d['r-vars.json']['ticker'] >>|, so let's now construct an R script to view and graph this time series in R. First, we load the necessary libraries:

<< d['time_series.R|idio|l']['libraries'] >>

Now, we simply call the FetchTimeSeries method, passing the ticker:

<< d['time_series.R|fn|idio|rint|pyg|l']['fetch-ts'] >>

We can easily plot the returned data:

<< d['time_series.R|fn|idio|rint|pyg|l']['plot-ts'] >>

\includegraphics{dexy--dbhts.pdf}

We can convert the time series to xts:

<< d['time_series.R|fn|idio|rint|pyg|l']['convert-xts'] >>

Description of as.xts.TimeSeries from documentation: << d['rdoc.R|rdoc']['OpenGamma']['as.xts.TimeSeries']['description'] >>

And plot as an xts object:

<< d['time_series.R|fn|idio|rint|pyg|l']['plot-xts'] >>

\includegraphics{dexy--dbhts-xts.pdf}


\section{Methods}

<% for k, v in d['rdoc.R|rdoc']['OG'].iteritems() -%>
\vspace{1cm}
\large
\textbf{<< k >>}
\normalsize

<< v['title'] >>

<< v['description'] >>

<% if v.has_key('arguments') -%>
\vspace{0.5cm}
\textbf{Arguments}

\fontshape{it}\selectfont
<% for arg_name, arg_desc in v['arguments'].iteritems() -%>
\verb|<< arg_name >>| << arg_desc >>

<% endfor -%>
\fontshape{n}\selectfont
<% endif -%>

<% endfor -%>

\end{document}

